% Options for packages loaded elsewhere
\PassOptionsToPackage{unicode}{hyperref}
\PassOptionsToPackage{hyphens}{url}
\PassOptionsToPackage{dvipsnames,svgnames,x11names}{xcolor}
%
\documentclass[
  letterpaper,
  DIV=11,
  numbers=noendperiod]{scrartcl}

\usepackage{amsmath,amssymb}
\usepackage{iftex}
\ifPDFTeX
  \usepackage[T1]{fontenc}
  \usepackage[utf8]{inputenc}
  \usepackage{textcomp} % provide euro and other symbols
\else % if luatex or xetex
  \usepackage{unicode-math}
  \defaultfontfeatures{Scale=MatchLowercase}
  \defaultfontfeatures[\rmfamily]{Ligatures=TeX,Scale=1}
\fi
\usepackage{lmodern}
\ifPDFTeX\else  
    % xetex/luatex font selection
\fi
% Use upquote if available, for straight quotes in verbatim environments
\IfFileExists{upquote.sty}{\usepackage{upquote}}{}
\IfFileExists{microtype.sty}{% use microtype if available
  \usepackage[]{microtype}
  \UseMicrotypeSet[protrusion]{basicmath} % disable protrusion for tt fonts
}{}
\makeatletter
\@ifundefined{KOMAClassName}{% if non-KOMA class
  \IfFileExists{parskip.sty}{%
    \usepackage{parskip}
  }{% else
    \setlength{\parindent}{0pt}
    \setlength{\parskip}{6pt plus 2pt minus 1pt}}
}{% if KOMA class
  \KOMAoptions{parskip=half}}
\makeatother
\usepackage{xcolor}
\setlength{\emergencystretch}{3em} % prevent overfull lines
\setcounter{secnumdepth}{-\maxdimen} % remove section numbering
% Make \paragraph and \subparagraph free-standing
\ifx\paragraph\undefined\else
  \let\oldparagraph\paragraph
  \renewcommand{\paragraph}[1]{\oldparagraph{#1}\mbox{}}
\fi
\ifx\subparagraph\undefined\else
  \let\oldsubparagraph\subparagraph
  \renewcommand{\subparagraph}[1]{\oldsubparagraph{#1}\mbox{}}
\fi

\usepackage{color}
\usepackage{fancyvrb}
\newcommand{\VerbBar}{|}
\newcommand{\VERB}{\Verb[commandchars=\\\{\}]}
\DefineVerbatimEnvironment{Highlighting}{Verbatim}{commandchars=\\\{\}}
% Add ',fontsize=\small' for more characters per line
\usepackage{framed}
\definecolor{shadecolor}{RGB}{241,243,245}
\newenvironment{Shaded}{\begin{snugshade}}{\end{snugshade}}
\newcommand{\AlertTok}[1]{\textcolor[rgb]{0.68,0.00,0.00}{#1}}
\newcommand{\AnnotationTok}[1]{\textcolor[rgb]{0.37,0.37,0.37}{#1}}
\newcommand{\AttributeTok}[1]{\textcolor[rgb]{0.40,0.45,0.13}{#1}}
\newcommand{\BaseNTok}[1]{\textcolor[rgb]{0.68,0.00,0.00}{#1}}
\newcommand{\BuiltInTok}[1]{\textcolor[rgb]{0.00,0.23,0.31}{#1}}
\newcommand{\CharTok}[1]{\textcolor[rgb]{0.13,0.47,0.30}{#1}}
\newcommand{\CommentTok}[1]{\textcolor[rgb]{0.37,0.37,0.37}{#1}}
\newcommand{\CommentVarTok}[1]{\textcolor[rgb]{0.37,0.37,0.37}{\textit{#1}}}
\newcommand{\ConstantTok}[1]{\textcolor[rgb]{0.56,0.35,0.01}{#1}}
\newcommand{\ControlFlowTok}[1]{\textcolor[rgb]{0.00,0.23,0.31}{#1}}
\newcommand{\DataTypeTok}[1]{\textcolor[rgb]{0.68,0.00,0.00}{#1}}
\newcommand{\DecValTok}[1]{\textcolor[rgb]{0.68,0.00,0.00}{#1}}
\newcommand{\DocumentationTok}[1]{\textcolor[rgb]{0.37,0.37,0.37}{\textit{#1}}}
\newcommand{\ErrorTok}[1]{\textcolor[rgb]{0.68,0.00,0.00}{#1}}
\newcommand{\ExtensionTok}[1]{\textcolor[rgb]{0.00,0.23,0.31}{#1}}
\newcommand{\FloatTok}[1]{\textcolor[rgb]{0.68,0.00,0.00}{#1}}
\newcommand{\FunctionTok}[1]{\textcolor[rgb]{0.28,0.35,0.67}{#1}}
\newcommand{\ImportTok}[1]{\textcolor[rgb]{0.00,0.46,0.62}{#1}}
\newcommand{\InformationTok}[1]{\textcolor[rgb]{0.37,0.37,0.37}{#1}}
\newcommand{\KeywordTok}[1]{\textcolor[rgb]{0.00,0.23,0.31}{#1}}
\newcommand{\NormalTok}[1]{\textcolor[rgb]{0.00,0.23,0.31}{#1}}
\newcommand{\OperatorTok}[1]{\textcolor[rgb]{0.37,0.37,0.37}{#1}}
\newcommand{\OtherTok}[1]{\textcolor[rgb]{0.00,0.23,0.31}{#1}}
\newcommand{\PreprocessorTok}[1]{\textcolor[rgb]{0.68,0.00,0.00}{#1}}
\newcommand{\RegionMarkerTok}[1]{\textcolor[rgb]{0.00,0.23,0.31}{#1}}
\newcommand{\SpecialCharTok}[1]{\textcolor[rgb]{0.37,0.37,0.37}{#1}}
\newcommand{\SpecialStringTok}[1]{\textcolor[rgb]{0.13,0.47,0.30}{#1}}
\newcommand{\StringTok}[1]{\textcolor[rgb]{0.13,0.47,0.30}{#1}}
\newcommand{\VariableTok}[1]{\textcolor[rgb]{0.07,0.07,0.07}{#1}}
\newcommand{\VerbatimStringTok}[1]{\textcolor[rgb]{0.13,0.47,0.30}{#1}}
\newcommand{\WarningTok}[1]{\textcolor[rgb]{0.37,0.37,0.37}{\textit{#1}}}

\providecommand{\tightlist}{%
  \setlength{\itemsep}{0pt}\setlength{\parskip}{0pt}}\usepackage{longtable,booktabs,array}
\usepackage{calc} % for calculating minipage widths
% Correct order of tables after \paragraph or \subparagraph
\usepackage{etoolbox}
\makeatletter
\patchcmd\longtable{\par}{\if@noskipsec\mbox{}\fi\par}{}{}
\makeatother
% Allow footnotes in longtable head/foot
\IfFileExists{footnotehyper.sty}{\usepackage{footnotehyper}}{\usepackage{footnote}}
\makesavenoteenv{longtable}
\usepackage{graphicx}
\makeatletter
\def\maxwidth{\ifdim\Gin@nat@width>\linewidth\linewidth\else\Gin@nat@width\fi}
\def\maxheight{\ifdim\Gin@nat@height>\textheight\textheight\else\Gin@nat@height\fi}
\makeatother
% Scale images if necessary, so that they will not overflow the page
% margins by default, and it is still possible to overwrite the defaults
% using explicit options in \includegraphics[width, height, ...]{}
\setkeys{Gin}{width=\maxwidth,height=\maxheight,keepaspectratio}
% Set default figure placement to htbp
\makeatletter
\def\fps@figure{htbp}
\makeatother

\KOMAoption{captions}{tableheading}
\makeatletter
\@ifpackageloaded{caption}{}{\usepackage{caption}}
\AtBeginDocument{%
\ifdefined\contentsname
  \renewcommand*\contentsname{Table of contents}
\else
  \newcommand\contentsname{Table of contents}
\fi
\ifdefined\listfigurename
  \renewcommand*\listfigurename{List of Figures}
\else
  \newcommand\listfigurename{List of Figures}
\fi
\ifdefined\listtablename
  \renewcommand*\listtablename{List of Tables}
\else
  \newcommand\listtablename{List of Tables}
\fi
\ifdefined\figurename
  \renewcommand*\figurename{Figure}
\else
  \newcommand\figurename{Figure}
\fi
\ifdefined\tablename
  \renewcommand*\tablename{Table}
\else
  \newcommand\tablename{Table}
\fi
}
\@ifpackageloaded{float}{}{\usepackage{float}}
\floatstyle{ruled}
\@ifundefined{c@chapter}{\newfloat{codelisting}{h}{lop}}{\newfloat{codelisting}{h}{lop}[chapter]}
\floatname{codelisting}{Listing}
\newcommand*\listoflistings{\listof{codelisting}{List of Listings}}
\makeatother
\makeatletter
\makeatother
\makeatletter
\@ifpackageloaded{caption}{}{\usepackage{caption}}
\@ifpackageloaded{subcaption}{}{\usepackage{subcaption}}
\makeatother
\ifLuaTeX
  \usepackage{selnolig}  % disable illegal ligatures
\fi
\usepackage{bookmark}

\IfFileExists{xurl.sty}{\usepackage{xurl}}{} % add URL line breaks if available
\urlstyle{same} % disable monospaced font for URLs
\hypersetup{
  pdftitle={CO2 emissions -r -research},
  colorlinks=true,
  linkcolor={blue},
  filecolor={Maroon},
  citecolor={Blue},
  urlcolor={Blue},
  pdfcreator={LaTeX via pandoc}}

\title{CO2 emissions -r -research}
\author{}
\date{}

\begin{document}
\maketitle

\subsection{Quarto}\label{quarto}

Quarto enables you to weave together content and executable code into a
finished document. To learn more about Quarto see
\url{https://quarto.org}.

\subsection{Running Code}\label{running-code}

When you click the \textbf{Render} button a document will be generated
that includes both content and the output of embedded code. You can
embed code like this:

\begin{Shaded}
\begin{Highlighting}[]
\CommentTok{\#Importing data from Csv the eission data}
\FunctionTok{library}\NormalTok{(readr)}
\NormalTok{emission\_data }\OtherTok{\textless{}{-}} \FunctionTok{read\_csv}\NormalTok{(}\StringTok{"C:/Users/knr/Desktop/R Bootcamp/Bootcamp{-}resource/Bootcamp{-}resource/Nomvume\_resource/data/emission\_data.csv"}\NormalTok{)}
\end{Highlighting}
\end{Shaded}

\begin{verbatim}
Rows: 132 Columns: 5
-- Column specification --------------------------------------------------------
Delimiter: ","
chr (3): entity, code, products
dbl (2): emission, per_capital_emission

i Use `spec()` to retrieve the full column specification for this data.
i Specify the column types or set `show_col_types = FALSE` to quiet this message.
\end{verbatim}

\begin{Shaded}
\begin{Highlighting}[]
\FunctionTok{View}\NormalTok{(emission\_data)}
\end{Highlighting}
\end{Shaded}

You can add options to executable code like this

\begin{Shaded}
\begin{Highlighting}[]
\CommentTok{\#converting emission from tonnes to kilo tonnes(1000 tons = 1kilo tonnes)}
\FunctionTok{library}\NormalTok{(dplyr)}
\end{Highlighting}
\end{Shaded}

\begin{verbatim}

Attaching package: 'dplyr'
\end{verbatim}

\begin{verbatim}
The following objects are masked from 'package:stats':

    filter, lag
\end{verbatim}

\begin{verbatim}
The following objects are masked from 'package:base':

    intersect, setdiff, setequal, union
\end{verbatim}

\begin{Shaded}
\begin{Highlighting}[]
\NormalTok{emission\_data }\SpecialCharTok{|\textgreater{}} 
  \FunctionTok{mutate}\NormalTok{(}
   \AttributeTok{kilo\_emission =}\NormalTok{emission}\SpecialCharTok{/}\DecValTok{1000}\NormalTok{)}
\end{Highlighting}
\end{Shaded}

\begin{verbatim}
# A tibble: 132 x 6
   entity    code  products       emission per_capital_emission kilo_emission
   <chr>     <chr> <chr>             <dbl>                <dbl>         <dbl>
 1 Australia AUS   Rice            879389.                0.524         879. 
 2 Australia AUS   Wheat            41497.                0.524          41.5
 3 Australia AUS   Other Cereals    89034.                0.524          89.0
 4 Austria   AUT   Rice            184118.                0.353         184. 
 5 Austria   AUT   Wheat            15495.                0.353          15.5
 6 Austria   AUT   Other Cereals    30146.                0.353          30.1
 7 Belgium   BEL   Rice            458813.                0.971         459. 
 8 Belgium   BEL   Wheat            50685.                0.971          50.7
 9 Belgium   BEL   Other Cereals    86200.                0.971          86.2
10 Brazil    BRA   Rice          10277208.                2.71        10277. 
# i 122 more rows
\end{verbatim}

\begin{Shaded}
\begin{Highlighting}[]
\CommentTok{\#Exploratory data analysis for emission\_data}
\CommentTok{\#head(data) represents the top 6 rows of the emission\_data quick snapshot}
\FunctionTok{head}\NormalTok{(emission\_data)}
\end{Highlighting}
\end{Shaded}

\begin{verbatim}
# A tibble: 6 x 5
  entity    code  products      emission per_capital_emission
  <chr>     <chr> <chr>            <dbl>                <dbl>
1 Australia AUS   Rice           879389.                0.524
2 Australia AUS   Wheat           41497.                0.524
3 Australia AUS   Other Cereals   89034.                0.524
4 Austria   AUT   Rice           184118.                0.353
5 Austria   AUT   Wheat           15495.                0.353
6 Austria   AUT   Other Cereals   30146.                0.353
\end{verbatim}

\begin{Shaded}
\begin{Highlighting}[]
\CommentTok{\#tail(data) represents the bottom 6 rows of the emission data quick snapshot}
\FunctionTok{tail}\NormalTok{(emission\_data)}
\end{Highlighting}
\end{Shaded}

\begin{verbatim}
# A tibble: 6 x 5
  entity         code  products      emission per_capital_emission
  <chr>          <chr> <chr>            <dbl>                <dbl>
1 United Kingdom GBR   Rice          1279479.                0.391
2 United Kingdom GBR   Wheat          119004.                0.391
3 United Kingdom GBR   Other Cereals  248065.                0.391
4 United States  USA   Rice          4564029.                0.347
5 United States  USA   Wheat          761895.                0.347
6 United States  USA   Other Cereals 1027924.                0.347
\end{verbatim}

\begin{Shaded}
\begin{Highlighting}[]
\CommentTok{\#summary provides overview of the data , five number summary(min,max,1st qartile , 3rd quartile , Range) }
\FunctionTok{summary}\NormalTok{(emission\_data)}
\end{Highlighting}
\end{Shaded}

\begin{verbatim}
    entity              code             products            emission       
 Length:132         Length:132         Length:132         Min.   :    1309  
 Class :character   Class :character   Class :character   1st Qu.:   17533  
 Mode  :character   Mode  :character   Mode  :character   Median :   79844  
                                                          Mean   :  933255  
                                                          3rd Qu.:  316456  
                                                          Max.   :62291319  
 per_capital_emission
 Min.   :0.06882     
 1st Qu.:0.20271     
 Median :0.38503     
 Mean   :0.50759     
 3rd Qu.:0.50162     
 Max.   :2.77977     
\end{verbatim}

\begin{Shaded}
\begin{Highlighting}[]
\CommentTok{\#provides a summary of the data frame’s columns}
\FunctionTok{str}\NormalTok{(emission\_data)}
\end{Highlighting}
\end{Shaded}

\begin{verbatim}
spc_tbl_ [132 x 5] (S3: spec_tbl_df/tbl_df/tbl/data.frame)
 $ entity              : chr [1:132] "Australia" "Australia" "Australia" "Austria" ...
 $ code                : chr [1:132] "AUS" "AUS" "AUS" "AUT" ...
 $ products            : chr [1:132] "Rice" "Wheat" "Other Cereals" "Rice" ...
 $ emission            : num [1:132] 879389 41497 89034 184118 15495 ...
 $ per_capital_emission: num [1:132] 0.524 0.524 0.524 0.353 0.353 ...
 - attr(*, "spec")=
  .. cols(
  ..   entity = col_character(),
  ..   code = col_character(),
  ..   products = col_character(),
  ..   emission = col_double(),
  ..   per_capital_emission = col_double()
  .. )
 - attr(*, "problems")=<externalptr> 
\end{verbatim}



\end{document}
